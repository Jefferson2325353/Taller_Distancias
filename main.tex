\documentclass[11pt,a4paper]{article}
\usepackage[spanish, es-nodecimaldot]{babel}
\usepackage[T1]{fontenc}
\usepackage[utf8]{inputenc}
\usepackage{lmodern}
\usepackage{amsmath, amssymb, amsthm, amsbsy, mathtools}
\usepackage{siunitx}
\usepackage{graphicx}
\graphicspath{{./}{figuras/}}

\usepackage{booktabs}
\usepackage{geometry}
\usepackage{hyperref}
\usepackage{enumitem}
\usepackage{csquotes}
\usepackage{physics}
\usepackage{multirow}
\usepackage{caption}
\usepackage{subcaption}
\usepackage{array}
\geometry{margin=2.5cm}
\setlength{\parskip}{6pt}
\setlength{\parindent}{0pt}
\sisetup{round-mode=places, round-precision=2, detect-weight=true, detect-family=true}

% ====== METADATOS (EDITAR) ======
\newcommand{\Curso}{Métodos Matemáticos para la Física I}
\newcommand{\Autor}{Nombre Apellido}
\newcommand{\Codigo}{Código}
\newcommand{\Fecha}{\today}

% ====== VALORES (EDITAR SI ES NECESARIO) ======
\newcommand{\PMcol}{PM2.5}     % nombre exacto de la columna en el Excel
\newcommand{\Wopt}{180}        % minutos
\newcommand{\DistMin}{497.44}
\newcommand{\AlphaGlobal}{0.5938}
\newcommand{\MAEglobal}{3.05}
\newcommand{\RMSEglobal}{3.83}
\newcommand{\AlphaTrain}{0.6352}
\newcommand{\MAEtrain}{3.49}
\newcommand{\RMSEtrain}{4.31}
\newcommand{\MAEtest}{2.68}
\newcommand{\RMSEtest}{3.45}
\newcommand{\TolAbs}{5}        % µg/m^3
\newcommand{\PctOK}{81.8}
\newcommand{\XvalidMin}{0.03}
\newcommand{\XvalidMax}{40.39}

% ====== RUTAS DE FIGURAS (COLOCAR ARCHIVOS EN LA MISMA CARPETA QUE ESTE .TEX) ======
\newcommand{\FigSerie}{serie_win180.png}
\newcommand{\FigCalib}{calibracion_win180.png}
% (Opcional) si generas la curva d(W) vs W:
\newcommand{\FigVentanaDist}{ventana_vs_distancia.png}
% (Opcional) residuos en el tiempo y histograma:
\newcommand{\FigResiduos}{residuos_tiempo.png}
\newcommand{\FigHistRes}{hist_residuos.png}

\title{\textbf{Taller de Distancias y Calibración Lineal de Sensores de PM\textsubscript{2.5}}\\
\large \Curso}
\author{\Autor\ (\Codigo)}
\date{\Fecha}

\begin{document}
\maketitle

\begin{abstract}
Este informe documenta un protocolo cuantitativo, reproducible y trazable para comparar y calibrar
lecturas de un sensor de bajo costo de material particulado fino (\textsc{pm}\textsubscript{2.5}) respecto de una estación de referencia (patrón).
El procedimiento sigue los lineamientos del \emph{Taller de Distancias}: (i) armonización temporal y construcción de promedios móviles sobre ventanas comunes, (ii) selección del ancho de ventana mediante la minimización de la distancia euclidiana entre series suavizadas, (iii) calibración lineal sin intercepto $f \approx \alpha\,\hat f$, (iv) validación temporal fuera de muestra, y (v) delimitación del alcance de validez bajo una tolerancia especificada.
Aplicado al conjunto de datos suministrado, se obtiene una ventana óptima de \SI{\Wopt}{\minute}, distancia mínima $d(W^\*)=\num{\DistMin}$ y un factor de calibración global $\alpha=\num{\AlphaGlobal}$. El desempeño agregado es $\mathrm{MAE}=\num{\MAEglobal}$ y $\mathrm{RMSE}=\num{\RMSEglobal}$; en validación mitad/mitad, $\alpha_{\text{train}}=\num{\AlphaTrain}$ con $\mathrm{MAE}_{\text{test}}=\num{\MAEtest}$ y $\mathrm{RMSE}_{\text{test}}=\num{\RMSEtest}$. Con tolerancia absoluta $\tau=\SI{\TolAbs}{\micro\gram\per\meter\cubed}$, el \num{\PctOK}\% de los puntos queda dentro de tolerancia y el rango operativo del sensor (suavizado) donde la calibración es confiable es $[\num{\XvalidMin}, \num{\XvalidMax}]$~\si{\micro\gram\per\meter\cubed}.
\end{abstract}

\section{Introducción y motivación}
La proliferación de sensores de bajo costo ha democratizado el monitoreo de calidad del aire; sin embargo, sus lecturas suelen exhibir ruido, sesgos sistemáticos, deriva y sensibilidad a variables ambientales.
Una \emph{calibración} frente a una estación de referencia es indispensable para convertir lecturas brutas en estimaciones útiles para análisis científico.
Este trabajo implementa un flujo \emph{end-to-end} de limpieza, alineación, suavizado, comparación, calibración, validación y reporte, con énfasis en reproducibilidad y trazabilidad, en el espíritu de la asignatura \Curso.

\section{Datos, supuestos y preprocesamiento}
\subsection{Descripción de los datos}
\begin{itemize}[leftmargin=1.2em]
  \item \textbf{Patrón (referencia):} columna \texttt{\PMcol} del archivo \texttt{Datos Estaciones AMB.xlsx}. La columna temporal presenta variantes (\texttt{Date\&Time}, \texttt{Fecha y Hora}, etc.).
  \item \textbf{Sensor (IoT):} colección de archivos \texttt{mediciones\_*.csv}, potencialmente con separadores y codificaciones heterogéneas.
\end{itemize}

\subsection{Normalización temporal y de tipo}
Para prevenir errores de fusión por zona horaria (\verb|datetime64[ns]| vs \verb|datetime64[ns,UTC]|), toda marca temporal $t$ se normaliza a \textbf{UTC naive} mediante el mapeo
\begin{equation}
t \mapsto \mathrm{NaiveUTC}(t) := \mathrm{tz\mbox{-}drop}\big(\mathrm{toUTC}(t)\big).
\end{equation}
Se convierten todas las magnitudes a tipo numérico, coercionando cadenas y descartando filas sin fecha o valor válidos.

\subsection{Grilla temporal común y suavizado}
Sea $\Delta t$ la mediana del paso del patrón; se remuestrean patrón y sensor por mediana (robusto a atípicos) sobre la grilla $t_k=t_0+k\,\Delta t$. Para cada $W$ (minutos) se define el \emph{promedio móvil centrado}:
\begin{equation}
\overline{f}_k(W)=\frac{1}{m(W)}\sum_{j\in \mathcal{W}_k(W)} f_j,\qquad
\overline{\hat f}_k(W)=\frac{1}{m(W)}\sum_{j\in \mathcal{W}_k(W)} \hat f_j,
\end{equation}
donde $\mathcal{W}_k(W)$ es la ventana centrada en $t_k$ e $m(W)$ el número de puntos en la ventana.
El centrado reduce desfases; $W$ controla el compromiso \emph{sesgo--varianza}.

\subsection{Emparejamiento temporal}
Se aplica una unión \emph{nearest} con tolerancia $\delta$ (típicamente 10--20~min) para obtener pares coetáneos $(\overline{\hat f}_i(W), \overline{f}_i(W))$. Esto preserva estructura temporal con pequeños desajustes.

\section{Selección de ventana por distancia euclidiana}
Para cada $W$ se evalúa
\begin{equation}
d(W)=\norm{\overline{\mathbf{f}}(W)-\overline{\hat{\mathbf{f}}}(W)}_2
=\sqrt{\sum_{i=1}^{n(W)} \big(\,\overline{f}_i(W)-\overline{\hat f}_i(W)\,\big)^2}.
\end{equation}
Se define $W^\*=\arg\min_W d(W)$. En los datos analizados: $W^\*=\SI{\Wopt}{\minute}$ con $d(W^\*)=\num{\DistMin}$. La Figura~\ref{fig:serie} ilustra la superposición suavizada en $W^\*$.

\begin{figure}[h]
\centering
\includegraphics[width=\textwidth]{\FigSerie}
\caption{Series suavizadas (patrón vs.\ sensor) con $W^\*=\SI{\Wopt}{\minute}$ y distancia mínima $d(W^\*)=\num{\DistMin}$.}
\label{fig:serie}
\end{figure}

\begin{figure}[h]
\centering
\includegraphics[width=0.7\textwidth]{\FigVentanaDist}
\caption{Curva $d(W)$ vs.\ $W$; la línea punteada marca $W^\*$.}
\label{fig:ventana}
\end{figure}

\section{Calibración lineal sin intercepto}
Se plantea $f \approx \alpha\,\hat f$ sin término independiente. Derivando la solución de mínimos cuadrados por el origen:
\begin{align}
\alpha^\* &= \arg\min_{\alpha}\, \sum_{i} \big(f_i - \alpha\,\hat f_i\big)^2
= \arg\min_{\alpha}\, \left( \sum_i f_i^2 - 2\alpha\sum_i f_i \hat f_i + \alpha^2\sum_i \hat f_i^2 \right),\\
\frac{d}{d\alpha}(\cdot)&=0 \;\Rightarrow\; -2\sum_i f_i\hat f_i + 2\alpha\sum_i \hat f_i^2 = 0
\;\Rightarrow\; \boxed{\;\alpha=\dfrac{\sum_i f_i\hat f_i}{\sum_i \hat f_i^2}\;}. \label{eq:alpha}
\end{align}
En nuestro caso, $\alpha_{\text{global}}=\num{\AlphaGlobal}$. La Figura~\ref{fig:calib} muestra la nube $(\overline{\hat f},\overline{f})$ y la recta $y=\alpha x$.

\begin{figure}[h]
\centering
\includegraphics[width=0.58\textwidth]{\FigCalib}
\caption{Dispersión $(\hat f, f)$ y recta $y=\alpha x$ con $\alpha=\num{\AlphaGlobal}$; MAE=\num{\MAEglobal}, RMSE=\num{\RMSEglobal}.}
\label{fig:calib}
\end{figure}

\subsection{Métricas de error}
Sea $e_i=f_i-\alpha\,\hat f_i$. Reportamos
\begin{equation}
\mathrm{MAE}=\frac{1}{n}\sum_i |e_i|,\qquad
\mathrm{RMSE}=\sqrt{\frac{1}{n}\sum_i e_i^2}.
\end{equation}
Con $\alpha$ global: $\mathrm{MAE}=\num{\MAEglobal}$ y $\mathrm{RMSE}=\num{\RMSEglobal}$, todo en \si{\micro\gram\per\meter\cubed}.

\section{Validación temporal (fuera de muestra)}
Se divide cronológicamente el conjunto en dos mitades: entrenamos $\alpha_{\text{train}}$ en la primera y evaluamos en la segunda. Se obtienen:
\begin{center}
\begin{tabular}{lccc}
\toprule
 & MAE & RMSE & Observaciones \\
\midrule
Global ($\alpha=\num{\AlphaGlobal}$) & \num{\MAEglobal} & \num{\RMSEglobal} & todas las fechas \\
Train ($\alpha_{\text{train}}=\num{\AlphaTrain}$) & \num{\MAEtrain} & \num{\RMSEtrain} & estima $\alpha$ \\
Test (con $\alpha_{\text{train}}$) & \num{\MAEtest} & \num{\RMSEtest} & generalización temporal \\
\bottomrule
\end{tabular}
\end{center}

\begin{figure}[h]
\centering
\includegraphics[width=\textwidth]{\FigResiduos}
\caption{Residuos en el tiempo $e(t)=f(t)-\alpha \hat f(t)$.}
\label{fig:residuos}
\end{figure}

\begin{figure}[h]
\centering
\includegraphics[width=0.55\textwidth]{\FigHistRes}
\caption{Histograma de residuos.}
\label{fig:histres}
\end{figure}

\section{Alcance de validez y tolerancias}
Con tolerancia absoluta $\tau=\SI{\TolAbs}{\micro\gram\per\meter\cubed}$, el \num{\PctOK}\% de los pares cumple $|e_i|\le \tau$.
Definimos el \emph{rango operativo} del sensor (suavizado) como
\begin{equation}
\big[\hat f_{\min}^{(\tau)},\, \hat f_{\max}^{(\tau)}\big]
= \text{mín./máx.\ de }\overline{\hat f}_i \text{ tales que } |e_i|\le\tau.
\end{equation}
En nuestro caso: $[\num{\XvalidMin},\,\num{\XvalidMax}]$~\si{\micro\gram\per\meter\cubed}. Si se desea una tolerancia relativa (porcentaje), reemplazar la condición por $|e_i|\le \rho\,|f_i|$.

\section{Análisis de sensibilidad y consideraciones prácticas}
\begin{itemize}[leftmargin=1.2em]
\item \textbf{Elección de $W$:} $W$ pequeño reduce sesgo y aumenta varianza; $W$ grande reduce varianza y aumenta sesgo (desfase/atenuación de picos). El óptimo hallado sugiere estructura de variabilidad de escala horaria (3~h).
\item \textbf{Estabilidad de $\alpha$:} comparar $\alpha$ global vs.\ $\alpha_{\text{train}}$ informa sobre deriva estacional o dependencia de humedad/temperatura.
\item \textbf{Grilla y tolerancia temporal:} una tolerancia de emparejamiento demasiado estricta elimina pares; demasiado laxa mezcla episodios disímiles.
\item \textbf{Robustez:} la mediana en el remuestreo mitiga atípicos frente a la media; se sugiere evaluar también Huber/biweight si hay outliers persistentes.
\end{itemize}

\section{Conclusiones}
El protocolo implementado provee una \textbf{calibración lineal simple y operacional} del sensor respecto del patrón, con una ventana óptima $W^\*=\SI{\Wopt}{\minute}$ que minimiza la discrepancia suavizada y un factor $\alpha=\num{\AlphaGlobal}$ aplicable en el rango operativo estimado. La validación temporal muestra consistencia en MAE/RMSE, lo que respalda el uso de $\alpha$ en periodos similares. Este flujo es reproducible e integrable en pipelines de monitoreo.

\section*{Reproducibilidad}
El notebook \texttt{TallerDistancias.ipynb} automatiza: carga (Excel/CSV), normalización temporal a UTC naive, remuestreo, ventana--distancia, ajuste de $\alpha$, validación, tolerancia y exportes (\texttt{figuras/}, \texttt{resultados/}). Este informe se compila con \LaTeX; basta colocar \texttt{\FigSerie} y \texttt{\FigCalib} en la misma carpeta.

\appendix
\section{Pseudocódigo del pipeline}
\begin{enumerate}[leftmargin=1.2em, label=\textbf{Paso \arabic*.}]
\item \textbf{Cargar patrón y sensor} (auto-detección de columnas de tiempo/valor).
\item \textbf{Normalizar tiempo} $\to$ UTC naive (ambas fuentes).
\item \textbf{Remuestrear} por mediana en grilla común $\Delta t = \mathrm{mediana}(\Delta t_{\text{patrón}})$.
\item Para cada $W$ en $\{15,30,60,120,180\}$~min:
  \begin{enumerate}
    \item Calcular promedios móviles centrados de patrón y sensor.
    \item Emparejar por tiempo (nearest, tolerancia $\delta$).
    \item Calcular $d(W)$.
  \end{enumerate}
\item Seleccionar $W^\*=\arg\min_W d(W)$.
\item Con $W^\*$: estimar $\alpha$ con \eqref{eq:alpha}; computar MAE/RMSE.
\item Validar mitad/mitad: $\alpha_{\text{train}}$ en la primera mitad; MAE/RMSE en la segunda.
\item Alcance de validez: porcentaje dentro de tolerancia y rango operativo.
\end{enumerate}

\section{Tabla de símbolos}
\begin{center}
\begin{tabular}{ll}
\toprule
Símbolo & Descripción \\
\midrule
$f$ & serie del patrón (referencia) \\
$\hat f$ & serie del sensor (bajo costo) \\
$W$ & ancho de ventana (minutos) \\
$d(W)$ & distancia euclidiana entre series suavizadas \\
$\alpha$ & factor de calibración (sin intercepto) \\
$e$ & residuo $f-\alpha\hat f$ \\
$\tau$ & tolerancia (absoluta o relativa) \\
\bottomrule
\end{tabular}
\end{center}

\end{document}
